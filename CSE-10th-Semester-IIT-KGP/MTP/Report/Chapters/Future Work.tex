\chapter{Future Work}
\label{Chapter8}
\lhead{Chapter 8. \emph{Future Work}}

\begin{center}
\rule{0.5\textwidth}{0.5pt}
\end{center}

This research establishes a foundation for agentic conversation generation in tobacco cessation, with several promising directions for future investigation. We identify key research opportunities that could advance both the theoretical understanding and practical applications of AI-driven healthcare interventions.

\section{Advanced Agent Architectures}

Future research should explore hierarchical frameworks with specialized clinical roles that mirror healthcare team structures. Formal verification methods for healthcare agent interactions would enhance safety and reliability in sensitive medical contexts. Investigation of emergent behaviors in multi-agent healthcare systems could reveal novel approaches to complex medical decision-making not possible with single-agent designs.

\section{Learning from Human Feedback}

Implementation of Reinforcement Learning from Human Feedback (RLHF) pipelines specifically optimized for tobacco cessation represents a promising direction for model improvement. Development of specialized reward functions that balance clinical accuracy with appropriate empathy would enhance therapeutic effectiveness. Comparative studies between RLHF and supervised approaches could identify optimal training methodologies for different healthcare scenarios.

\section{Cross-domain Knowledge Transfer}

Systematic evaluation of model transfer capabilities to other addiction contexts would establish the generalizability of the approach. Quantification of domain-specific knowledge retention during transfer learning could inform efficient adaptation strategies. Development of robust metrics for measuring generalization across medical specialties would advance the broader application of these techniques in healthcare.

\section{Longitudinal Conversation Modeling}

Research into architectures capable of modeling long-term patient-provider relationships is essential for chronic condition management. Temporal attention mechanisms that track patient progress over extended periods could enhance personalization and intervention timing. Creation of evaluation frameworks specifically designed to assess conversation coherence over time would better reflect real-world healthcare interactions.

\section{Technical Optimization}

Optimization of inference latency for real-time conversation would improve user experience in mobile healthcare applications. Techniques for reducing computational requirements would enable deployment in resource-constrained environments typical of many healthcare settings. Development of evaluation metrics that correlate with clinical outcomes rather than purely linguistic measures would better align technical performance with healthcare objectives.

\section{Privacy and Explainability}

Implementation of differential privacy guarantees for synthetic conversation generation would address critical healthcare data protection requirements. Federated learning approaches could enable model improvement while protecting sensitive patient information. Integration of clinical knowledge graphs with chain-of-thought reasoning would enhance the transparency and clinical validity of AI-generated recommendations.

The integration of these research directions with rigorous clinical validation would significantly advance AI-assisted healthcare interventions, extending benefits beyond tobacco cessation to other behavioral health domains requiring personalized, longitudinal support.
